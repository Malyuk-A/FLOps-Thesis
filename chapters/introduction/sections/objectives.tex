\section{Objectives}\label{section:objectives}

The motivation allows the following key objectives for such a tool to emerge.
\vspace{5mm}
\newline
\textbf{Improve Accessibility}\newline
Making FL more accessible by abstracting away and automating complexities enables further individuals to engage with it.
Expanding FL to more areas will increase its usage and user base, raising general interest and relevance for its field, which should aid its development.
\vspace{5mm}
\newline
\textbf{Benefit from Automation}\newline
Automating tedious, error-prone, and repetitive manual tasks necessary to perform FL will save time and resources for critical work that advances the field.
\vspace{5mm}
\newline
\textbf{Prioritize Practical FL Application}\newline
This tool should focus on being usable in real physical conditions on distributed devices.
FL struggles with a gap between research/virtual-simulation and practical application in real production environments.
It should be feasible to incorporate this tool into existing workflows.
\vspace{5mm}
\newline
\textbf{Embrace Flexibility}\newline
Because FL is such a young and active field, it faces constant change.
This tool should welcome change in the form of extendability and adaptability.
It should be flexible and applicable to a multitude of use cases and scenarios.
This tool should be easy to modify to accommodate evolving needs.
It should profit from existing technologies to offer a higher level of quality than creating everything from the grounds up.
