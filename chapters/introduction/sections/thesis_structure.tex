\section{Thesis Structure}

The following background chapter discusses vital concepts required to understand FLOps as a whole and in detail.
It covers fundamental and detailed aspects of FL, such as the basic process, various architectures, frameworks, and tools.
It analyses the research field of FL in great detail.
The background chapter continues to showcase essential FLOps parts, including ML operations and orchestration.
The concluding background section highlights related work.

Chapter three performs requirements engineering and showcases significant parts of FLOps' system design.
Firstly, it elicits and specifies functional and nonfunctional requirements.
Secondly, it discusses system models that depict the FLOps system and satisfy these requirements.
This chapter aims to provide a comprehensive understanding of the system through its requirements and simplified architecture and processes without delving into underlying technicalities.

The fourth chapter analyses concrete implementation details.
It starts by explaining how users can request workloads via FLOps' API and SLAs.
It elaborates on the image-building processes in FLOps.
It delves into the rationale, challenges, and internal solutions for building suitable containerized images.
Afterward, this chapter explains how local data is managed on the learning worker nodes.
Next, it showcases the architecture of its MLOps components and how its GUI works.
The penultimate section of this chapter discusses how FLOps realizes clustered hierarchical FL.
The last section discusses the accompanying new CLI and how it makes working with FLOps and its orchestrator easier.

The evaluation chapter assesses the soundness and performance of FLOps.
It compares different setups, experimental configurations, and their results.

The final chapter draws conclusions about the current FLOps project and implementation.
It discloses FLOps' limitations and alludes to possible future work to improve the system further.

