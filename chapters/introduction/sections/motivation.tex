\section{Motivation}

Building or contributing to an FL framework or library focusing on the previously mentioned challenges could soften or entirely alleviate those problems. 
Such a tool should have Docker and Kubernetes as role models regarding their mature features to improve application-oriented practical workflows and accessibility.
It should strive to be comparable to them but for the discipline of FL.
It should specialize in the setup, deployment, component management, and automation, in short, FL orchestration.
Allowing researchers, developers, and end-users to set up, perform, reproduce, and experiment with FL in a more accessible way.
The goal of this tool should be to automate and simplify complex tasks, reducing the required level of expertise in various domains.
These areas range from ML/FL, dependency management, containerization technologies, and orchestration to automation.
This tool would streamline and accelerate existing workflows and future progress by utilizing reliable automation to avoid error-prone manual tasks.
With its potential to optimize, standardize, and unify processes, this envisioned tool could become a significant part of the emerging FL ecosystem.
Such a tool could enable less experienced people to perform FL, engage with, and contribute to the field of FL.
As a result, these techniques could improve the entire discipline of FL, and more individuals in more areas could access and benefit from FL.