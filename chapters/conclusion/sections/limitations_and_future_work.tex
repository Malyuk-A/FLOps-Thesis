\section{Current Status, Limitations \& Future Work}

Table \ref{table:status_functional_reqs} depicts how well the current FLOps implementation fulfills its functional requirements.
We count FR-1.3 and FR-2 as MVP implemented because FLOps supports different FL scenarios, such as classic and clustered HFL, as well as various project configurations.
Additional FL strategies, algorithms, and configurations exist that should be analyzed and added to future FLOps versions to allow a broader range of FL capabilities.
FLOps currently supports Scikit-learn, PyTorch, and TensorFlow.
Other ML flavors can be added by extending the underlying minimal enum structure.
We regard FR-6 as MVP implemented because enabling inference serving was not a primary focus point of FLOps.
Inference serving is FLOps's last post-training step.
Thus, we did not invest a lot of time into this feature.
It should be thoroughly tested and investigated to be seen as fully realized.
MLflow offers different ways of turning trained models into inference servers \cite{mlflow_inference_serving}.
Comparing these variations and letting users decide how to create inference servers is a valid choice for future work.

% https://aneescraftsmanship.com/circle-symbols%E2%97%8B%E2%97%8F%E2%97%8D%E2%97%97%E2%97%94%E2%97%99%E2%A6%BF-in-latex/
\begin{itemize}
    \item [\faCircleO] \textbf{Not Implemented}: FLOps does not meet the requirement in its current form. 
	\item [\faDotCircleO] \textbf{Partially Implemented}: FLOps only partially meets the requirement.
	\item [\faArrowCircleRight] \textbf{Implemented MVP}: FLOps fulfills the requirement in a minimal viable way.
	\item [\faCircle] \textbf{Fully Implemented}: FLOps fully realizes the requirement.
\end{itemize}

\begin{figure}[p]
	\begin{table} [H]
    \begin{center}
    \begin{tabular}{l  r} 
    \hline
    Functional Requirement & Current Status  \\ [0.5ex] 
    \hline
    FR-1.1 Enable individuals to use, develop, and evaluate practical FL & \faCircle  \\
    FR-1.2 Automate FL management \& processes & \faCircle  \\
    FR-1.3 Support various flexible FL scenarios & \faArrowCircleRight  \\
    FR-2 Provide flexible configuration & \faArrowCircleRight  \\
    FR-3 Handle FL augmentation and containerization & \faCircle  \\
    FR-4 Provide a GUI for monitoring, evaluation, and result management & \faCircle  \\
    FR-5 Provide trained model access to users & \faCircle  \\
    FR-6 Enable inference serving & \faArrowCircleRight  \\
    \hline 
    \end{tabular}
    \end{center}
    \caption{Status of Functional Requirement Fulfillment}
    \label{table:status_functional_reqs}
\end{table}
	\begin{table} [H]
  \begin{center}
  \begin{tabular}{l  r} 
  \hline
  Nonfunctional Requirement & Current Status  \\ [0.5ex] 
  \hline\hline
    \multicolumn{2}{c}{Usability} \\ 
  \hline
  NFR-1.1 Effortless FL Participation & \faCircle  \\
  NFR-1.2 Prepared Reusable Component & \faCircle  \\
  NFR-1.3 GUI & \faCircle  \\
  NFR-1.4 CLI & \faCircle  \\
  \hline 
    \multicolumn{2}{c}{Maintainability} \\
  \hline
  NFR-2.1.1 Codebase & \faCircle  \\
  NFR-2.1.2 Quality Enforcement & \faCircle  \\
  \hline 
    \multicolumn{2}{c}{Portability} \\
  \hline
  NFR-2.2.1 ARM \& AMD Support & \faCircle  \\
  NFR-2.2.2 Generic Interfaces & \faDotCircleO  \\
  \hline 
    \multicolumn{2}{c}{Performance} \\
  \hline
  NFR-3.1 Scalability & \faArrowCircleRight  \\
  NFR-3.2 Availability & \faDotCircleO \\
  NFR-3.2.1 Error Handling & \faCircle  \\
  NFR-3.3 Optimized Image Building & \faCircle  \\
  \hline 
    \multicolumn{2}{c}{Security} \\
  \hline
  NFR-4 Security & \faCircleO  \\
  \hline 
    \multicolumn{2}{c}{Constraints} \\
  \hline
  NFR-5.1 Packaging & \faCircle  \\
  NFR-5.2 Implementation & \faCircle  \\
  \hline
  \end{tabular}
  \end{center}
  \caption{Status of Nonfunctional Requirement Fulfillment}
  \label{table:status_nonfunctional_reqs}
\end{table}	
\end{figure}

Table \ref{table:status_nonfunctional_reqs} shows the grade of fulfillment of nonfunctional requirements in the current FLOps version.
FLOps only partially realizes NFR-2.2.2.
During the development of FLOps, we had to introduce several minor changes and additions to Oakestra so that it could handle FLOps workflows.
The FLOps project structure follows Oakestra's application and service structure, which differs from that of other orchestrators like Kubernetes.
Due to the restricted time, we did not try to run FLOps via another orchestrator.
Therefore, we need to find out how straightforward using FLOps with other orchestrators is.
Nonetheless, FLOps communicates with Oakestra via decoupled APIs and SLAs.
Replacing or extending these interaction points to other orchestrators should be straightforward.
In addition, Jabok Kempter's work combining Oakestra and Kubernetes \cite{thesis:tum_jakob_kempter_kubernetes_oakestra} alludes that FLOps can run on Kubernetes as an Oakestra addon.

We marked NFR-3.1 as MVP implemented and NFR-3.2 as partially implemented because FLOps can run on monolithic and small multi-cluster setups with varying FL actors.
We have yet to try to run FLOps on a truly large scale, such as several hundred or thousand devices.
Therefore, we cannot indefinitely confirm that It can handle large-scale deployments.
Regarding availability, FLOps includes several mechanisms of communication, error and event handling.
However, if FLOps services or applications are modified or deleted by an external non-FLOps source, current FLOps is incapable of reacting accordingly.
We hope to change this once FLOps is connected to the new Oakestra event hooks \cite{thesis:tum_mahmoud} so it can respond to and receive these vital events and react accordingly.


\subsubsection{Implementation}
In addition to using existing libraries and frameworks, FLOps includes many custom components we implemented using various state-of-the-art tools and techniques from the ground up.
During this work, we used many of these tools for the first time, with no prior experience or training in them.
Thus, some aspects of the implementation might utilize these tools subparly.
FLOps' codebase is a one-man project that lacks code reviews from others, especially domain/tool experts.
To improve robustness and code quality, individuals with expertise in these domains should review the code and elevate it to follow the gold standards of these tools.

\subsubsection{Image Building}
FLOps uses a sophisticated image-building architecture and processes, yet evaluations show that this aspect takes up a significant portion of project run times and lead to large images.
New experimental solutions should be explored to improve the build times further and shrink the final images.
One way would be to replace the current Conda-Mamba solution with uv \cite{uv}.
Other tools worth exploring to slim down images include \cite{slim,dragonfly,nydus}.

\subsubsection{Security \& Privacy}
Security and privacy are FL's foundational concerns.
Our priority with FLOps was to create and verify its rudimentary architecture, components, and workflows.
We did not focus on privacy and security concerns due to the lack of access to proper certificates and to accelerate FLOps' development.
This includes using HTTP instead of HTTPS or not supporting FL security features such as secure aggregation.
This aspect is one of the most important for future work.
Notably, Flower \cite{flower_tls}, the image registry \cite{docs:cncf_distribution_registry}, and the local data management's Arrow Flight \cite{docs:arrow_flight} all have security features.
FLOps can utilize these features by enabling and appropriately configuring them.

\subsubsection{Federated Learning via FLOps}
FLOps already enables classic and clustered HFL.
Many other possible improvements should be considered for future work.
Possible directions include exploring and adding FL native security and privacy mechanisms or extending the already available functionalities by adding more parameters and different algorithms.
So far, FLOps has not focused on GPU workloads, which are supported by Flower and should thus already be able to run via FLOps.
FLOps should be tried out with ML frameworks specifically targeted for resource-restricted edge and IoT devices.
These experiments should be evaluated on edge devices such as Raspberry Pis and hybrid setups.

One significant aspect that should be investigated is personalized FL.
For classic PFL, the trained global model would be deployed on every learner, trained further for local use, and then used for inference serving on the learner nodes.
The sidecar or multi-model PFL solution could look like this.
The global model would be trained as is, but a personal model would also be trained concurrently and not shared or updated with other learners.
Once the training is done, the local model can be turned into an inference server.
Hybrids between both are also possible.
One way to realize these in FLOps is to augment or extend the learner services so that they hold a local model, persist after training, and transform into inference servers, or that the FLOps manager replaces them via inference servers instead.
Our reading contained multiple interesting PFL papers (\ref{subsection:fl_research}), so we think this direction is very promising and worth working on.

\subsubsection{Complementary Components \& Integrations}
To work with FLOps, users must clone its repository, do slight configurations, and launch its components.
Its CLI could be further improved to automate even more aspects of FLOps and its orchestrator.
Ideally, the CLI could install all the needed dependencies, repositories, and components of FLOps with a single command.
This includes installing its orchestrator and launching its components to make working with these tools as easy and quick as possible.
Realizing this goal is straightforward because the CLI already has many necessary functionalities that can be expanded and combined.
FLOps should be appropriately integrated into Oakestra's new addon marketplace created by Mahmoud Elkodary \cite{thesis:tum_mahmoud}.
In addition FLOps should use the new Oakestra hooks to react to events.
Once this is realized, FLOps will be even more responsive and accessible for users.

