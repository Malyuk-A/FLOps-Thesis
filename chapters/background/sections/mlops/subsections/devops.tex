
\subsection{DevOps}

The history of software development shows an evolution from static and isolated to dynamic and intertwined workflows.
Older methods like the waterfall model split up the development and operations tasks and involved individuals.
Software was first developed by one team and then operated by another.
There was a massive increase in modern requirements for flexibility and the ability to change.
Developmental and operational tasks now form an interconnected infinite loop.
For example, a company develops the first version of a software product in-house.
They build distributable software artifacts based on their source code for distribution among clients.
These artifacts might be container images or executable binaries.
They publish these artifacts to online registries and roll live services out in the cloud.
Users enjoy this product and request further features.
The loop starts anew.
The new features lead to unexpected bugs.
The loop starts again, and so on.
A software loop is only as fast as its slowest step.

In today's world, software development loops are rarely linear sets of steps.
Such loops are running in parallel at different stages several times per day.
This concurrency is especially noticeable in projects that divide software into multiple decoupled parts.
For example, in micro-service architectures, one service might be buggy and need fixing while another receives a feature update.
These dynamic and strong dependencies require developmental and operational tasks to work tightly together.
This coupling also applies to IT professionals who must cooperate and understand each other's areas well.
This combined effort has become its own broad disciple called DevOps.

The synergy between development and operations created new techniques, tools, and professions.
This combination includes various tasks such as building, deploying, testing, and monitoring.
Automation is one core activity in this connected discipline because repetitive manual labor is an inefficient and expensive bottleneck.
Prominent tools include Ansible and Gitlab-CI/CD.
DevOps is a very broad discipline without concrete borders.
Building artifacts or container images, orchestration, or knowledge sharing can all be considered part of DevOps.
This notion makes Git, Docker, and Kubernetes the primary tools in this field.

An essential concept in DevOps is CI/CD, which stands for continuous integration, continuous delivery, and deployment.
CI/CD focused on automating this software loop via custom pipelines.
A DevOps pipeline is comparable to an assembly line in a factory.
A software product needs to pass several connected stages with multiple steps.
These stages can include testing, building, releasing, and deployment.

DevOps as a term was first mentioned around 2009 \cite{paper:mlops}.
This field is still very active and rapidly evolving.
Unfortunately, many other disciplines are not taking inspiration from or taking advantage of DevOps.
