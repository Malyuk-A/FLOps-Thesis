\subsection{FL Frameworks \& Libraries}\label{subsection:fl_frameworks_and_libraries}

\begin{figure}[b]
    \begin{changemargin}{0cm}{0cm}
    \centering
    \begin{tabular}{|m{2.4cm}||c|c|c|c|c|c|}
        \hline
            \textbf{Framework} & \textbf{Version} & \textbf{Releases in 2024} & \textbf{Stars} & \textbf{Forks} & \textbf{Open Issues} \\
        \hline
            Pysyft \cite{fl_framework:pysyft} & 0.9.0 & 62 & 9.4k & 2k & 2
        \\
        \hline
            Tensorflow Federated \cite{fl_framework:tensorflow_federated} & 0.85.0 & 18 & 2.3k & 578 & 168
        \\
        \hline
            FedML \cite{fl_framework:fedml} & 0.8.9 & 0 & 4.1k & 776 & 118
        \\
        \hline
            Flower \cite{fl_framework:flower} & 1.10.0 & 4 & 4.7k & 815 & 284
        \\
        \hline
            OpenFL \cite{fl_framework:openfl} & 9.3.4 & 2 & 1.9k & 426 & 256
        \\
        \hline
    \end{tabular}
    \captionof{table}{Updated FL Framework Comparison} 
    \label{table:updated_fl_framework_comparison}
\end{changemargin}

\end{figure}

This subsection examines the current landscape of available FL frameworks.
The goal is to better comprehend why most researchers did not specify or use FL frameworks in the previous findings.
This discussion will be brief because Saidani already analyzed and evaluated FL frameworks in great detail in his master's thesis \cite{thesis:tum_fl_framework_comparison} from 2023.
He examined FL libraries, frameworks, and benchmarks.
Many FL tools exist for specific niche use cases and architectures \cite{thesis:tum_fl_framework_comparison}.
This finding contradicts the opinions of his questioned FL practitioners and experts.
They expected FL libraries and frameworks to focus on basic FL features, such as communication, aggregator-learner orchestration, security, and data aggregation.
Many libraries and frameworks, most of which were not production-ready, were still in an experimental research state \cite{thesis:tum_fl_framework_comparison}.

To reduce complexity, he focused on the five most promising open-source frameworks.
For a framework to be allegeable, it had to fulfill 2/3 of the following criteria.
It needed more than one thousand starts and 350 forks on GitHub.
The interviewed experts had to mention it.
The framework had to support all major operation systems.
Because FL is rapidly evolving, we updated his findings and expanded upon them.
We included the number of releases (including patches) in 2024 so far and the number of open issues in the repository.

Table \ref{table:updated_fl_framework_comparison} shows the updated FL Framework comparison (16.08.2024).
These FL frameworks are in active development. 
Only FedML has not been updated for several months now.

Saidani's main original contribution was a novel FL benchmarking suite called FMLB (Federated Machine Learning Benchmark).
Its goal was to evaluate and compare the mentioned FL frameworks efficiently.
His previous analysis and summary of existing frameworks were sound and helpful.
However, we are critical of his evaluation results, especially the poor performance of Flower.
We tried to replicate his experiments, but his provided code \cite{tum_fl_framework_thesis_github} lacks instructions on how to set up this benchmark application.
We simulated the experiments with the latest official flower version at the time and made sure to stick as closely as possible to the same experimental setup and configuration.
Our findings show very different results.
Flower manages to solve the experiment quickly and efficiently.
Our results match the verdicts of other works comparing FL frameworks, such as \cite{comparative_analysis_of_fl_frameworks} or \cite{comprehensive_study_fl_frameworks_degree_project}.
\cite{comparative_analysis_of_fl_frameworks} is the latest work that compares FL frameworks that we considered, and its verdict is that Flower outperforms all its competition.
FLOps uses Flower as its FL framework of choice.