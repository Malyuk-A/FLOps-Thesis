
\section{Rationale}

FLOps is a new system connecting various custom and pre-made components that power a large set of features and functionalities.
This plethora of interconnected parts enables a large number of possible combinations that each could be evaluated.
We look at the SLA examples from \ref{subsection:SLAs} and formalize configuration parameters as variables.
\vspace{5mm}
\newline
\textbf{Variables defining a FLOps Project}
\begin{itemize}
    \item [R]:
        The ML Repository.
        Includes arbitrary complex and different ML code.
    \item [F]:
        The ML Model Flavor / Framework.
    \item [D]:
        The dataset that is used for training and evaluation.
    \item [B]:
        A boolean value indicating if base images should be used to speed up builds during development.
    \item [P]:
        The number of supported platforms.
        Currently FLOps supports two different platforms, which can be used together or in isolation.
        Thus, P offers three different options.
    \item [H]:
        The boolean flag indicating if FLOps should use classic or hierarchical FL.
    \item [T]:
        The number of training rounds or cycles.
    \item [L]:
        The number of required learners.
    \item [S]:
        The underlying setup of devices.
        We worked with a monolith and multi-cluster setup.
\end{itemize}

Even if we greatly simplify and assume that all these variables have only two possible assignments, we get an overwhelming number of combinations.
Each of these variables are independent of each other.
Therefore, each simplified boolean variable combined results in $ 2^9 = 512$ unique experiments.
We even omit several additional possible variables, such as running these tests on ARM or AMD devices, including Raspberry PIs, or using GPU workloads.
We aim for representative results, so every experiment is repeated ten times.
We call such repeated experiments evaluation cycles.
Such cycles take half an hour to several hours to complete.
This long runtime, combined with the total number of possible permutations, makes it infeasible to test every single combination.
This is a classic case of the curse of dimensionality, where increasing numbers of features result in an exponential increase of their feature space.

We focus on a manageable subset of different experiments that represent comprehensive combinations of these variables to cover a wide range of possible scenarios.
In addition, FLOps is a foundational new system.
Thus, we aim to verify and evaluate whether it works rather than how competitive it is against modern benchmarks and state-of-the-art training results.
We focus on the general workflow and the necessary steps to power such processes instead of optimizing FL training results.
Tables \ref{table:chosen_experiments_part_1} and \ref{table:chosen_experiments_part_2} depict our selected experiments and our reasons for choosing them.
The tinted cells highlight the critical changes between experiments.
Small classic FL experiments use two learners for three training rounds, and large classic FL experiments use four learners with ten learning rounds.
Small HFL experiments use two learners per cluster for three training rounds and two training cycles.
In total each learner trains for six rounds.
We explicitly try to change only singular variables to facilitate comparisons and understand how individual variables impact the whole workflow.

\begin{figure}[p]
    \begin{changemargin}{-2.75cm}{0cm}
    \centering
    \begin{tabular}{|M{0.35cm}|M{2cm}||m{5.5cm}|c|M{1.4cm}|c|c|c|c|M{0.9cm}|}
        \hline
            \textbf{ID} & \textbf{Experiment} & \textbf{Rationale} & \textbf{Setup} & \textbf{Data} & \textbf{ML} & \textbf{T+L} & \textbf{HFL} & \textbf{B} & \textbf{P} \\
        \hline
            1 &
            Base Case - Simplest &
            Represents the base case for comparisons and references.
            If it does not work nothing else should.
            Should be as simple and lightweight as possible.
            &
            Monolith &
            MNIST &
            Sklearn &
            Small &
            No &
            No &
            AMD
        \\
        \hline
            2 &
            Simple Large &
            Same as (1) but with more training rounds and learners.
            Checks if different, larger project sizes work.
            Focuses on the impact of project sizes.
            &
            Monolith &
            MNIST &
            Sklearn &
            \cellcolor{blue!15} Large &
            No &
            No &
            AMD
        \\
        \hline
            3 &
            Simple Base-Images &
            Same as (1) but uses development BaseImages to speed up builds.
            Verifies that using base images leads to the same end results, thus they are viable option for development.
            Checks the build acceleration due to BaseImages.
            &
            Monolith &
            MNIST &
            Sklearn &
            Small &
            No &
            \cellcolor{blue!15} Yes &
            AMD
        \\
        \hline
            4 &
            Simple Multi-Platform &
            Same as (1) but supports multiple target platforms.
            Verifies that FLOps supports multi-platform images.
            Checks the impact on build times.
            &
            Monolith &
            MNIST &
            Sklearn &
            Small &
            No &
            No &
            \cellcolor{blue!15}  AMD + ARM
        \\
        \hline
            5 &
            Simple Pytorch &
            Similar to (1) but uses a different ML repository, framework, and dataset.
            Verifies that FLOps supports various ML repositories, frameworks, and datasets.
            &
            Monolith &
            \cellcolor{blue!15} CIFAR10 &
            \cellcolor{blue!15} Pytorch &
            Small &
            No &
            No &
            AMD
        \\
        \hline
    \end{tabular}
    \captionof{table}{FLOps Evaluation Experiments I} 
    \label{table:chosen_experiments_part_1}
\end{changemargin}
\end{figure}

\begin{figure}[H]
    \begin{changemargin}{-2.75cm}{0cm}
    \centering
    \begin{tabular}{|M{0.35cm}|M{2cm}||m{5.5cm}|c|M{1.4cm}|c|c|c|c|M{0.9cm}|}
        \hline
            \textbf{ID} & \textbf{Experiment} & \textbf{Rationale} & \textbf{Setup} & \textbf{Data} & \textbf{ML} & \textbf{T+L} & \textbf{HFL} & \textbf{B} & \textbf{P} \\
        \hline
            6 &
            Simple HFL &
            Same as (1) but uses HFL.
            Verifies that HFL works in a controlled environment on a single machine with a single cluster.
            The base case for HFL.
            If this case does not work no other HFL should work.
            &
            Monolith &
            MNIST &
            Sklearn &
            Small &
            \cellcolor{blue!15} Yes &
            No &
            AMD
        \\
        \hline
            7 &
            Simple Multi-Cluster &
            Same as (1) but on the the multi-cluster setup.
            Verifies that the base case works on the other setup.
            If it should not work no other experiments make sense on that setup.
            &
            \cellcolor{blue!15} Multi-Cluster &
            MNIST &
            Sklearn &
            Small &
            No &
            No &
            AMD
        \\
        \hline
            8 &
            % Simple HFL on Multi-Cluster &
            HFL on Multi-Cluster &
            Same as (7) but using HFL.
            Verifies that real clustered HFL works as intended on multiple clusters.
            &
            \cellcolor{blue!15} Multi-Cluster &
            MNIST &
            Sklearn &
            Small &
            \cellcolor{blue!15} Yes &
            No &
            AMD
        \\
        \hline
    \end{tabular}
    \captionof{table}{FLOps Evaluation Experiments II} 
    \label{table:chosen_experiments_part_2}
\end{changemargin}
\end{figure}