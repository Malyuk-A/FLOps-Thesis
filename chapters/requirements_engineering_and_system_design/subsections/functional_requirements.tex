\subsection{Functional Requirements}
Functional Requirements (FR) describe mandatory functional relationships between the proposed system and its surroundings.
These requirements only focus on concrete functionalities, user interactions, and environmental conditions.
They ignore implementation details and nonfunctional conditions, such as performance.
FRs capture what a proposed system must achieve instead of how it achieves it. \cite{book:bruegge}

\begin{itemize}
    \item [FR-1] {\textbf{Federated Learning}} \label{FR-1}
        \begin{itemize}
        \item [FR-1.1] \textbf{Enable individuals to use, develop, and evaluate practical FL}: \label{FR-1.1}
            FLOps allows users with different level of expertise to utilize FL.
            Target groups include inexperienced and expert users, developers, and researchers.
        \item [FR-1.2] \textbf{Automate FL management \& processes}: \label{FR-1.2}
            FLOps automatically handles all necessary duties to perform FL for the user.
            These duties include providing, creating, (un)deploying, and removing FL components, such as learners and aggregator(s).
            FLOps starts and stops the training and evaluation processes.
        \item [FR-1.3] \textbf{Support various flexible FL scenarios}: \label{FR-1.3}
            Besides classic FL, FLOps supports (clustered) HFL.
            FLOps is ML library/framework agnostic, allowing different ML techniques, such as DNNs and classic ML.
        \end{itemize}
    \item [FR-2] {\textbf{Provide flexible configuration}}: \label{FR-2}
        FLOps supports different FL project configurations.
        For example, users can specify and request different resource requirements, such as the number of training rounds, FL algorithms, and the minimum number of learners.
    \item [FR-3] {\textbf{Handle FL augmentation and containerization}}: \label{FR-3}
        FLOps automatically converts user ML code into FL-capable container images that include all necessary dependencies to do FL.
        It stores these images internally and deploys them for training.
    \item [FR-4] {\textbf{Provide a GUI for monitoring, evaluation, and result management}}: \label{FR-4}
        FLOps provides a sophisticated GUI for monitoring, comparing, storing, exporting, sharing, and organizing training runs, metrics, and trained models.
        Users can access this GUI at any time.
        They can follow live training progress or inspect their previous results.
    \item [FR-5] {\textbf{Provide trained model access to users}}: \label{FR-5}
        FLOps enables users to access their trained models.
        FLOps can build container images that serve the trained models.
        Users can pull these images to use their trained models as they wish.
        Users can see and access their models via the GUI.
    \item [FR-6] {\textbf{Enable inference serving}}: \label{FR-6}
        FLOps can automatically build and deploy inference servers based on trained models.
        Users can send inference requests to their trained models directly after training on the same platform.
\end{itemize}