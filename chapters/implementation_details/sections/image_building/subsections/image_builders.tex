\subsection{Image Builders} \label{subsection:image_builders}

The most popular and widespread containerization tool is Docker.
It is also preferred when building images.
Usually, building images via Docker is straightforward.
One creates a Dockerfile and runs Docker commands to build a new image on a host machine based on this file.

FLOps requires a more complex approach to build images.
Its image builder is a service running on a worker node that Oakestra orchestrates.
This service is a containerd container.
It is not possible to simply run docker in such an environment to build images.
The critical issue here is that docker cannot be run trivially in another container due to restricted privileges and access to host system capabilities \cite{paper:dind}.
A popular workaround, especially for CI/CD workloads, is called Docker in Docker (DinD) \cite{paper:dind}.
Many CI/CD tools support DinD natively.
One example includes GitLab's CI/CD.
Setting up and using DinD manually on a host machine is non-trivial and error-prone.
Getting docker to work inside managed Oakestra containers is even more challenging.
One significant reason for these issues is that docker requires its daemon to run for building images.
Nested Docker daemons are a complicated matter \cite{paper:dind} that can require significant adjustments to how the orchestrator coordinates containerization on worker nodes.
For example, the orchestrator could require worker nodes to mount their docker sockets to the container, which would lead to further security risks.

The only functionality that FLOps requires here is building images inside containers, not executing them from within.
There exist various alternatives to Docker that specialize in building containerized images.
Thanks to the coordinated efforts of the OCI \cite{open_container_initiative} and others, these tools all build compatible images.
Alternative tools include kaniko, Podman \cite{docs:podman}, and Buildah \cite{buildah_homepage}.
FLOps uses Buildah.

Buildah is an open-source daemon-free tool that specializes in building OCI-compliant images.
It can run on host machines or inside containers.
It features many equivalent Docker commands and supports building images step-wise, programmatically, or via Dockerfiles.
Red Hat released Buildah initially in 2017, and its first major version was released in 2018.
Podman and Buildah are complementary tools that have some shared documentation and aspects.
Great resources to find out more about Podman, Buildah, and their relationship are available here \cite{redhat_docs:podman,redhat_docs:buildah,buildah_vs_podman}.

It was challenging to make container-internal image-building work for FLOps via Oakestra.
This image build process is especially complicated because it performs heavy dependency resolutions and installations.
We needed to add a new optional flag to Oakestra deployments that enabled to mount /dev/fuse.
The FUSE userspace filesystem framework enables non-privileged and secure mounts \cite{docs:fuse_linux_kernel}.
In addition, Buildah has to build its images with the chroot isolation flag enabled.
As a result, FLOps uses a highly fine-tuned and optimized builder environment that enables it to build images inside containers of orchestrated worker nodes.

Furthermore, we put a lot of thought and time into refining and polishing FLOps' docker files.
They build the foundation for all FLOps components, from project services and management components to auxiliary containers.