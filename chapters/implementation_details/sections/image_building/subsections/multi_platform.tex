\subsection{Multi-Platform}

This subsection explains how FLOps builds multi-platform images.
Usually, end users build images only for a single platform.
Their builder knows their host's architecture and uses it as a target platform.
When inspecting popular public images, they support multiple target platforms.
Each image tag can have multiple digests, each representing a different platform.
For example, the latest Alpine image \cite{alpine_multiplatform_image} supports at least linux/amd64 and linux/arm/v6.
By default, it is impossible to run images that are intended for different architectures on machines with incompatible host architectures.
Specific workarounds via emulation exist.

The key of building and referencing images that support multiple platforms are manifests.
A plain manifest file contains information about a unique image digest.
This information includes its media type, size, layers, and architecture.
When an image supports multiple platforms, it has multiple digests, thus one manifest per digest.
Image indexes were invented to group these different manifests.
Note that image indexes refer to the OCI standard term \cite{oci_image_index}.
In Docker, they are called fat manifests or manifest lists \cite{docs:docker_manifest}.
As a result, different host architectures can use the same image tag and pull their matching digest image.
This happens because the local builder reads the image index and picks a suitable manifest.
Examples for manifests are available here \cite{docs:docker_manifest}.

Multi-platform images are a deep topic.
Because of its rapid development, many different media types, versions, and schemas for manifests exist.
Build machines need to use the same conventions as their image registries.
Discussing these details here would lead to bloat.
Excellent information about this topic is available here \cite{docs:image_manifest_versions_schemas}.

Previously, one had to manually build one image per target architecture on a machine of that architecture, provide the image tag with an architecture suffix, and push it.
Once all these different images were pushed, an image index had to be created and pushed.
Nowadays, a convenient solution for building multi-platform images is using docker's buildx builder \cite{docs:docker_buildx}.
It can build and push multi-platform images concurrently with a single command.

The FLOps image builder does not use docker to build its images (\ref{subsection:image_builders}).
Buildah also supports building multi-platform images but lacks the convenient new features of docker's buildx.
From our experience, Buildah lacks sufficient documentation regarding building and pushing multi-platform images.
We needed to look into its source code, read other sources, and experiment extensively until we made this work.

Another significant requirement for building multi-platform images on a single machine is emulating other architectures.
Otherwise, only images for the host's architecture can be built.
It is also possible to cross-compile or use multiple dedicated builder nodes with proper architectures \cite{docs:docker_multiplatform_image_builds}.
Conventionally, QEMU \cite{docs:quemu} is used to emulate such tasks.
Docker Desktop includes QEMU for Mac and Windows by default \cite{docs:docker_multiplatform_image_builds}.
QEMU translates the requested target architecture instructions into ones the host machine can understand.
Due to FLOps' special circumstances, which involve building complex images in orchestrated, containerd containers on heterogeneous devices, various approaches seem to be available to realize emulation.
Ideally, the emulator would be part of the builder image, thus avoiding the need to modify or require anything from the worker nodes.
After many unsuccessful attempts, we decided to require worker nodes that should build multi-platform images to pre-install QEMU.
For Linux machines, this can be done by installing an open-source package called qemu-user-static.

In conclusion, using this approach, the FLOps builder service can build multi-platform images.
All other FLOps (static) images, including the builder service or project observer, are also available for linux/amd64 and linux/arm64.
For example, all these images can be started on a Raspberry Pi 4, but these devices seem to lack sufficient resources to handle the FL training.
An additional downside of multi-platform image builds comes with the slowness of emulation.

Table \ref{table:sklearn_mnist_multi_platform_build_example} shows a simple example of FLOps' builder service's build times for different platforms.
The build machine natively supports linux/amd64.
Therefore, this build is much faster than the emulated arm one.
Both rows show the build times when only a single platform is requested.
The build times would be combined to support both platforms simultaneously.

\begin{figure}[t]
    \begin{changemargin}{0cm}{0cm}
        \centering
        \begin{tabular}{|c|c|c|c|c|c|c|}
            \hline
            \textbf{Builder Architecture} & \textbf{Target Platform} & \textbf{Full Build} & \textbf{Base Image} & \textbf{Actor Images} \\
            \hline
            linux/amd64 & linux/amd64 & 4min & 2min 30s & 1min 30s \\
            \hline
            linux/amd64 & linux/arm64 & 18min & 12min & 6min \\
            \hline
        \end{tabular}
        \captionof{table}{Simple Scikit-learn Multi-Platform Build Times Example} 
        \label{table:sklearn_mnist_multi_platform_build_example}
    \end{changemargin}
\end{figure}