\section{User Interactions with the FLOps Manager}
This section details how users can interact with the FLOps manager.
It shows the currently available API endpoints and the SLA structure.

\subsection{API} \label{subsection:api}
We implemented the FLOps manager API via the Flask-OpenAPI3 framework \cite{framework:flask_openapi3}.
It is a relatively young and small framework that aligns Flask with OpenAPI3.
It uses Pydantic to verify data and automatically generate REST API and OpenAPI documentation that is compatible with popular frameworks such as Swagger.
We use Waitress \cite{waitress}, a production-quality WSGI server, to host the manager's API.

\subsubsection{Currently Available Endpoints}
\texttt{/api/flops/projects}\newline
This POST endpoint triggers a new FLOps project.
It expects users to provide a JSON SLA with the required project configurations and a bearer token authorizing the user on the orchestrator.
If no matching images exist, an image builder is created and deployed.
If an adequate image already exists, the request concludes straight away.
The user receives a confirmation that the new project has successfully started.
\vspace{5mm}
\newline
\texttt{/api/flops/tracking}\newline
The tracking endpoint allows users to spawn their personal tracking servers at will independently from an active project.
Usually, a tracking server is created during FL training.
This GET endpoint returns the tracking server / GUI URL.
\vspace{5mm}
\newline
\texttt{/api/flops/database}\newline
This DELETE endpoint only allows admins to reset the FLOps database.
Otherwise, the entire FLOps management suite needs a restart.
It returns a confirmation for the user.
\vspace{5mm}
\newline
\texttt{/api/flops/mocks}\newline
This POST endpoint creates mock data providers and deploys them on fitting learner machines.
We discuss these data providers later on in this thesis.
Similar to the project, this endpoint returns a confirmation to the user.
\vspace{5mm}
\newline

\pagebreak
\subsection{SLAs} \label{subsection:SLAs}
The FLOps manager can only instantiate a new project via an SLA.
This service layer agreement currently has the following structure.

% \begin{lstlisting}[language=json]
\begin{lstlisting}
    {
        % This key enables more verbose logging in the manager and project observer.
        "verbose": true, % default=false, optional
        % This ID should be the same as for the orchestrator.
        "customerID": "Admin", 
        % FLOps has only been tested for GitHub so far.
        "ml_repo_url": "https://github.com/Malyuk-A/flops_ml_repo_mnist_sklearn",
        % Supported flavors include: sklearn, pytorch, tensorflow, keras.
        "ml_model_flavor": "sklearn",
        % This key only works for special repositories intended for development.
        % It tells the builder to use prebuilt base images to significantly speed up image builds and development.
        "use_devel_base_images": false, % default=false, optional
        % This key expects a list of target platforms on which the built images should run.
        % It supports linux/amd64 and linux/arm64.
        "supported_platforms": ["linux/amd64"], % default=["linux/amd64"], optional
        "training_configuration": {
            % This key specifies the FL type.
            % FLOps supports classic and hierarchical modes.
            "mode": "classic", % default="classic", optional
            % Requested data tags should match available data on learner nodes.
            % Multiple different tags can be requested.
            % The ML data server will use local data fragments that match any of the provided tags.
            % If no data tags are provided the learner will notify watchers that it cannot find any data.
            "data_tags": ["mnist"],
            % Training cycles only apply to the hierarchical mode.
            "training_cycles: 1, % default=1, optional
            % This key tells learners the number of training and evaluation rounds to perform.
            "training_rounds": 3, % default=3, optional
            % Clients mean learners in this context.
            "min_available_clients": 2, % default=1, optional
            "min_fit_clients": 2, % default=1, optional
            "min_evaluate_clients": 2 % default=1, optional
        },
        % FLOps supports these two post-training steps.
        "post_training_steps": ["build_image_for_trained_model", "deploy_trained_model_image"], % default=[], optional
        % These are optional values that require fine-tuning.
        % Note that these values are not recommendations but placeholder values.
        "resource_constraints": {
            "memory": 250, % in MB
            "vcpus": 1,
            "storage": 25 % in MB
        }
    }
\end{lstlisting}

The following SLA shows a simple classic FL project configuration with both post-training steps enabled.
It requires two learners for training.

\begin{lstlisting}[language=json]
    {
        "verbose": true,
        "customerID": "Admin",
        "ml_repo_url": "https://github.com/Malyuk-A/flops_ml_repo_mnist_sklearn",
        "ml_model_flavor": "sklearn",
        "training_configuration": {
            "data_tags": ["mnist"],
            "min_available_clients": 2,
            "min_fit_clients": 2,
            "min_evaluate_clients": 2
        },
        "post_training_steps": ["build_image_for_trained_model", "deploy_trained_model_image"],
    }
\end{lstlisting}

The next SLA displays a more advanced configuration with HFL, multi-platform support, and more FL actors.

\begin{lstlisting}[language=json]
    {
        "customerID": "Admin",
        "ml_repo_url": "https://github.com/Malyuk-A/flops_ml_repo_cifar10_pytorch",
        "ml_model_flavor": "pytorch",
        "supported_platforms": ["linux/amd64", "linux/arm64"],
        "training_configuration": {
            "mode": "hierarchical",
            "data_tags": ["cifar10"],
            "training_cycles": 10,
            "training_rounds": 5,
            "min_available_clients":3,
            "min_fit_clients": 3,
            "min_evaluate_clients": 3
        },
        "post_training_steps": ["build_image_for_trained_model", "deploy_trained_model_image"],
    }
\end{lstlisting}