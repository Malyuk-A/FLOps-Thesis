\subsection{ML \& Big Data Formats}

The field of Big Data and ML formats is vast and complex.
It provides many insights into different optimization approaches.
Great resources to find out more are available here \cite{influxdb_apache_stack,apache_arrow_flight_sql_2022,columnar_roadmap_parquet_and_arrow,apache_arrow_flight_intro}.
The following are significant takeaways after investigating this domain.
\vspace{5mm}
\newline
\textbf{Managing Big Data is a booming Field}\newline
Storing and handling Big Data is a massively popular, expensive, and profitable business that regularly attracts hundred-million-dollar investments.
This environment leads to healthy competition, solid standards, and bold advancements in the field.
\vspace{5mm}
\newline
\textbf{Reuse}\newline
Data management and optimizations are universally needed.
These areas have several decades of solid research to back up best practices and avoid known pitfalls.
Similarly to security, one should avoid reinventing and reimplementing foundational features from the ground up.
Instead, it is recommended that existing open-source industry-favored solutions be reused.
\vspace{5mm}
\newline
\textbf{(De)Serialization is a critical Bottleneck}\newline
When each subsystem has its own internal memory format, significant amounts of CPU work get wasted on (de)serialization.
The recommended way to avoid this is to stick to a uniform format.
The more tools support such a uniform format, the easier and faster cooperation, communication, and transmission can be.
\vspace{5mm}
\newline
\textbf{Big Data and ML Data should use Columnar Formats}\newline
Usually, dataset features are split up into different columns.
These features can be diverse and require different data types for storage.
When storing and handling conventionally stored row-wise data, all these different data types and features complicate and slow down processing.
Instead, if the data is stored and handled column-wise, advanced optimizations, and compressions can handle homogeneous features and their data type.
As a result, the same data can be processed more compactly and faster.
\vspace{5mm}
\newline
The sources above recommend the following open source state-of-the-art formats and technologies.
All of them are from Apache.
\vspace{5mm}
\newline
\textbf{Arrow}\newline
Arrow is a language-agnostic columnar memory format.
It is optimized for modern CPUs and GPUs.
It supports zero-copy reads, which avoid serialization and accelerate data access.
Arrow is especially popular for interoperability.
This format should be used for data in memory. \cite{arrow_repo} 
\vspace{5mm}
\newline
\textbf{Parquet}\newline
Parquet is also a columnar file format.
Its focus is on efficient data storage and retrieval.
Its benefit over Arrow is that it needs less space due to its special compression and encoding.
This format should be used to store data on disks. \cite{docs:parquet}
\vspace{5mm}
\newline
\textbf{Arrow Flight}\newline
Arrow Flight is a gRPC-based framework.
It supports parallel data streams.
When used with compatible data, it overcomes (de)serialization overheads and speeds up data transfer.
Flight should be used to transport Apache formatted data over the network. \cite{docs:arrow_flight}
\vspace{5mm}
\newline
The last subsection mentioned that FLOps uses Flower Datasets, which uses Hugging Face Datasets underneath.
Hugging Face Datasets use Arrow \cite{docs:hugging_face_arrow}.
Therefore, the findings in this subsection are directly applicable and relevant to FLOps.