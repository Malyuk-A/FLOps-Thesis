\section{CLI}

While developing FLOps, we implemented several different pieces of code to automate tedious, repetitive manual tasks.
We decided to share and offer these custom auxiliary scripts by combining them into a single CLI tool called OAK CLI \cite{cli_code}.
We have steadily improved it over this work, which resulted in various features.
Because of the envisioned rapid changes in the future, we will not discuss concrete technical details but provide a broad overview of its capabilities.
We decided to discuss this CLI as part of this work because FLOps can be considered DevOps for FL, and DevOps also includes techniques to improve development workflows.
One way to support users and developers is to help them use the target application more conveniently, for example, via a CLI tool.
Notably, Oakestra has a custom early-stage work-in-progress GUI/Frontend dashboard application \cite{oakestra_dashboard} and a minimal CLI tool.
The OAK CLI is independent of both these components and replaced the legacy Oakestra CLI tool.

We were motivated to create this CLI to alleviate the following challenges.
FLOps does not offer a tool besides the OAK CLI to interact conveniently with its API.
External tools like Postman are necessary to do so.
Oakestra components need to be prepared and launched manually.
Interacting with its API is possible via external tools or its early-stage dashboard.
We needed to work on several devices while developing FLOps, especially its HFL features.
Each machine must be appropriately configured and set up to enable Oakestra workloads.
This setup included installing dependencies such as Docker, Golang, and other custom aliases and scripts to launch, clear, and restart Oakestra components.
In addition, developing and observing FLOps was heavily bound to the observatory features provided by Oakestra's dashboard application.
Accessing and working with this dashboard was cumbersome even during local development on a single machine because of repetitive click-based tasks such as mandatory logins.
We automated this by creating auto-clicker scripts via Selenium.
Accessing this flaky dashboard on remote devices behind firewalls required even more manual steps, such as code modifications and multiple SSH tunnels.
We only required the dashboard to get an overview of the deployed application and services and their statuses and logs.
The OAK CLI satisfies these needs independently, bypassing our need for the dashboard while automating manual steps.
As a result, this CLI significantly accelerates the development and usage of FLOps and Oakestra.

\subsection{CLI Requirements Discussion}

The OAK CLI needs to satisfy the following requirements:
\vspace{5mm}
\newline
\textbf{Interface for APIs}\newline
The CLI should interact with the APIs of FLOps and Oakestra to alleviate the need for users to use external tools, know all API endpoints, and know how to interact with them appropriately.
The key activities the CLI should support are managing applications and services in Oakestra and starting projects in FLOps.
For example, if the user wants to create a new application in Oakestra he first needs to login.
For the login, the user needs to know the login URI, create a fitting request, send it, and extract the received bearer token for authentication and authorization.
Only afterward can users prepare their application SLA, add their token, and send it to the application POST endpoint.
Instead, the CLI should offer a single command so that users only need to provide their application SLA. 
The CLI will perform the login and handle the API interactions.
\vspace{5mm}
\newline
\textbf{Observability}\newline
FLOps handles many different applications and services concurrently.
Especially during development, it is crucial to observe whether components behave as expected and identify unexpected errors or behavior as quickly as possible.
For this, the CLI needs to provide its users with an overview of the current state of applications and deployed services.
This overview should include comprehensive information about these components, such as their current status, critical properties, and details.
It is crucial to provide timely information to the user.
The CLI should support a way to observe the current situation close to real-time.
\vspace{5mm}
\newline
\textbf{Accelerated Workflows via Automation}\newline
The more manual, tedious, repetitive tasks can be accelerated via automation, the more high-quality work can be done and less frustration generated.
The tool should provide ways of installing dependencies to make Oakestra's and FLOps' setup quicker and easier on new machines.
The CLI should handle starting, stopping, restarting, and rebuilding Oakestra and FLOps components to speed up and simplify development cycles.
In addition, it should be a common place to host valuable additions from different individuals, including aliases or scripts.
The CLI should be easily modified and extended to accompany future user demands.
\vspace{5mm}
\newline
Based on these requirements, it is easy to see that this singular tool has many responsibilities to uphold and features to offer.
These features are of no equal interest to all its possible users.
Casual end-users require and demand other features than administrators or developers.
The usable feature set is also heavily dependent on the machine on which the CLI runs.
A machine can be a standalone control plane that only includes the root or cluster orchestrators.
It can be a standalone worker node or a hybrid of several scenarios.
A machine can also simultaneously include all these mentioned components and be a monolith system.
All these options require different needs and, conversely, do not require all features or are not capable of or intended to provide all features.
For example, a worker node should be unable to restart the root orchestrator, whereas the root orchestrator should not be able to tinker with sensitive worker node configurations.

A tool that fulfills all mentioned requirements can be divided into several, one for each use case or provided by a dynamic large single tool.
Multiple tools have the benefit of being less overwhelming, leading to smaller and tidier repositories, and users do not need to understand the big picture.
The downside of multiple tools is the risk of entangled and divergent dependencies that need to be managed.
Such tools will still show interdependencies and coupling, leading to split comprehension.
Users, especially developers, would need to know exactly what tool is responsible for what and how they are interconnected, potentially making things more complex.
With split tools, there is the risk of increased code duplication over time, especially if different people develop different parts without understanding related tools.
The benefit of a single tool is that everything is in one place and forms a single source of truth.
Developers can efficiently work with a single well-structured repository.
Users only need to install and update a single tool instead of several.
Code and logic can be more easily reused between the same code base.
The downsides of a single tool are the risk of high coupling and increased complexities as the project grows.
OAK CLI is a single tool because we explicitly structured it to support low coupling and high cohesion.
Its flexible design enables easy separation and extension.
Thus, all FLOps-specific functionalities can be easily isolated and converted into a separate CLI tool.
We aim towards a modular monolith design that suits a smaller team like ours.

\subsection{High Level CLI Details}

The CLI can create, delete, and (un)deploy applications and services.
It can present current information about available apps and services in different verbosity levels.
In addition, it allows to inspect latest service logs.
Users can use the CLI to manage FLOps components, including its manager, database, projects, and tracking server.
The CLI automates tedious manual tasks such as clearing local images or rebuilding orchestrator components.
It can install fundamental dependencies such as Docker and Golang.
The CLI has a customizable config file that allows users to specify their intended use case and filter out unwanted commands.
It also features experimental commands to run automatic evaluations.
Detailed descriptions of the CLI commands and options with screenshots are available in the appendix \ref{appendix:cli}.